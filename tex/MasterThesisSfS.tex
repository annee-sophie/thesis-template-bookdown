%%%--- Template for master thesis at SfS
%%%--- Modified template with more comments and examples -- SG, 11/06/09
%%%------
\documentclass[11pt,a4paper,twoside,openright]{report}
%%not needed \usepackage{E}
\usepackage[utf8]{inputenc}
\usepackage[english]{style/ETHDAsfs}%--> ETHDASA + fancyheadings + ... "umlaute"
%  + sfs-hyper -> hyperref

\usepackage{pdfpages}%%to include the confirmation of originality (plagiarism
\usepackage{amsbsy}%% for \boldsymbol and \pmb{.}
\usepackage{amssymb}%% calls  amsfonts...
\usepackage{booktabs} % For \toprule etc. in tables
\usepackage[table]{xcolor} 
%or \usepackage{german8}%-- =  german  +  isolatin1
\usepackage{graphicx}%-- f?r PostScript-Grafiken (besser als  psfig!)
%\usepackage[draft]{graphicx} % grafics shown as boxes --> faster compilation
%
\usepackage[longnamesfirst]{natbib}%was {sfsbib}%- F?r  Literatur-Referenzen
%           ^^^^^^^^^^^^^^ 1) "Hampel, Ronchetti, ..,"  2) "Hampel et al"
% Engineers (and other funny people) want to see [1], [2]
% ---> use 'numbers' : \usepackage[longnamesfirst,number]{natbib}
%
%
\usepackage{style/texab}%- 'tex Abk?rzungen' /u/sfs/tex/tex/latex/texab.sty
        %%- z.B.  \R, \Z, \Q, \Nat f?r reelle, ganze, rationale, nat?rl. Zahlen;
        %%-       \N   (Normalvert.)  \W == Wahrscheinlichkeit .....
        %%-  \med, \var, \Cov, \....
        %%-  \abs{x} == |x|   und   \norm{y} ==  || y ||   (aber anst?ndig)
%% NOTE: texab contains many useful definitions and "shortcuts". It is
%% worth to open the file and have a look at them. HOWEVER, some
%% definitions can lead to conflicts with other packages. You
%% might for example want to comment out the line defininf \IF as an
%% operator when working with the algorithmic package, or to comment out
%% the line defining a command \Cite with working with the Biblatex package
\usepackage{amsmath}
%\usepackage{mathrsfs}% Raph Smith's Formal Script font --> provides \mathscr
\usepackage{enumerate}% Fuer selbstdefinierte Nummerierungen
\usepackage{longtable}
%--------
\usepackage{relsize}%-> \smaller (etc) used here
\usepackage{color} %% to allow cloring in code listings
\usepackage{listings}% Fuer R-code, C-code, ....  and settings for these:
\definecolor{Mygrey}{gray}{0.75}% for linenumbers only!
\definecolor{Cgrey}{gray}{0.4}% for comments
\lstloadlanguages{R}
\lstset{ %% Hilfe unter z.B. http://en.wikibooks.org/wiki/LaTeX/Packages/Listings
language=R,
basicstyle=\ttfamily\scriptsize,%%- \small > \footnotesize > \scriptsize > \tiny
%commentstyle=\ttfamily\color{Cgrey},
commentstyle=\itshape\color{Cgrey},
numbers=left,
numberstyle=\ttfamily\color{Mygrey}\tiny,
stepnumber=1,
numbersep=5pt,
backgroundcolor=\color{white},
showspaces=false,
showstringspaces=false,
showtabs=false,
frame=single,
tabsize=2,
captionpos=b,
breaklines=true,
%breakatwhitespace=false,
keywordstyle={},
morekeywords={},
xleftmargin=4ex,
literate={<-}{{$$\leftarrow$$}}1 {~}{{$$\sim$$}}1}
\lstset{escapeinside={(*}{*)}} % for (*\ref{ }*) inside lstlistings (Scode)
%%----------------------------------------------------------------------------

%%------- Theoreme ---
\newtheorem{definition}{Definition}[subsection]
\newtheorem{lemma}[definition]{Lemma}
\newtheorem{theorem}[definition]{Theorem}
\newtheorem{Coro}[definition]{Corollary}
\theoremstyle{definition}
\newtheorem{example}[definition]{Example}
\newtheorem*{note}{Note}
\newtheorem*{remark}{Remark}

\DeclareMathOperator*{\plim}{plim}
% \def\MR#1{\href{http://www.ams.org/mathscinet-getitem?mr=#1}{MR#1}}

% \newcommand{\Lecture}[3]{\marginpar{#3.#2.#1}}
% \newcommand{\Fu}{\mathcal{F}}
\newcommand{\aatop}[2]{\genfrac{}{}{0pt}{}{#1}{#2}}

%\renewcommand{\theequation}{\arabic{equation}}
\numberwithin{equation}{subsection}

%%%%%%%%%%%%%%%%%%%%%%%%%%%%%%%%%%%%%%%%%%%%%%%%%
%%% Path for your figures                      %%%
%%%%%%%%%%%%%%%%%%%%%%%%%%%%%%%%%%%%%%%%%%%%%%%%%
% Set the paths where all figures are taken from:
\graphicspath{{images/}}

%%%%%%%%%%%%%%%%%%%%%%%%%%%%%%%%%%%%%%%%%%%%%%%%%
%%% Define your own commands here             %%%
%%%%%%%%%%%%%%%%%%%%%%%%%%%%%%%%%%%%%%%%%%%%%%%%%
\newcommand{\Bruch}[2]{{}^{#1}\!\!/\!_{#2}}
\renewcommand{\labelenumi}{\roman{enumi}.)}
\providecommand{\tightlist}{%
  \setlength{\itemsep}{0pt}\setlength{\parskip}{0pt}}

$header-includes$
$if(highlighting-macros)$
$highlighting-macros$
$endif$

\begin{document}
\bibliographystyle{style/chicago}% ---> Hampel,F., E.Ronchetti,... W.Stahel(1986) ...
 %was \bibliographystyle{sfsbib}\citationstyle{dcu} %OR DEFAULT : \citationstyle{agsm}

\pagenumbering{roman}%- roman numbering for first few pages

%%%%%%%%%%%%%%%%%%%%%%%%%%%%%%%%%%%%%%%%%%%%%%%%%
%%% Title page                                %%%
%%%%%%%%%%%%%%%%%%%%%%%%%%%%%%%%%%%%%%%%%%%%%%%%%
\period{$date$}
\dasatype{Master Thesis}
\students{$author$}
$if(coadviser)$
\alternatereaderprefix{Co-Adviser:}
\alternatereader{$coadviser$}
\mainreaderprefix{Adviser:}
\mainreader{$adviser$}
$else$
\alternatereaderprefix{Adviser:}
\alternatereader{$adviser$}
\mainreaderprefix{}
\mainreader{}
$endif$


\submissiondate{$submission_date$}
\title{$title$}

\maketitle%- Titelseite wird abgeschlossen
\cleardoublepage
 %%~~~~~~~~~~~~~~~~~~~~~~~~~~~~~~~~~~~~~~~~

%%%%%%%%%%%%%%%%%%%%%%%%%%%%%%%%%%%%%%%%%%%%%%%%%
%%% Insert here acknowledgements and abstract %%%
%%%%%%%%%%%%%%%%%%%%%%%%%%%%%%%%%%%%%%%%%%%%%%%%%
%% Dedication (optional)
\markright{}
\vspace*{\stretch{1}}
\begin{center}
    To some special person
\end{center}
\vspace*{\stretch{2}}

% Preface (optional)
\newpage
\markboth{Preface}{Preface}
\include{tex/Preface}

% Abstract should not be longer than one page.
\newpage
\markboth{Abstract}{Abstract}
\include{tex/Abstract}


\newpage
\SweaveInput{rnw/Sweave.rnw}

\newpage
\markboth{Test}{Test}

\begin{document}
\maketitle 
\subsection*{Introduction}
This section is based on "The Analysis of Covariance and Alternatives: Statistical Methods for Experiments, Quasi‐Experiments, and Single‐Case Studies, Second Edition"

The main advantages of including the covariate in randomized experiments are: 
\begin{enumerate}
  \item generally greater power (in RCT is this the major payoff)
  \item a reduction in bias caused by chance differences between groups that exist before the treatments are administered (this bias is generally small in randomized designs) 
  \item conditionally unbiased estimates of treatment effects 
\end{enumerate}

Types of covariates: 
\begin{itemize}
  \item baseline measures 
  \item organismic characteristics 
  \item environmental characteristics 
\end{itemize}

\subsubsection*{Conditional versus Unconditional Inference}
In expectation, random assignement provides groups that are exactly equivalent. So it follows that the difference between sample  means on Y (after treatment) is an unbiased estimate of the treatment effect because the groups are in expecation the same the only difference between the groups is due to the treatment.\\

But in the case of a single experiment the two groups will never be exactly the same on a continuous variable before treatments are applied. This means if we use the difference between sample means on Y gives us a conditionally biased estimate of the treatment effect because groups were not exactly the same before treatment. This does not mean that an ANOVA F-test and the associated effect estimate are wrong, but it is possible to do better.

ANCOVA incorporates the information available on the X variable and provides a conditionally unbiased (conditional on X) estimate of the treatment effect.\\

The long run average of these two types of effect estimates are the same under random assignement, but the ANCOVA estimates are more precise. 

\subsubsection*{General Ideas Associated with ANCOVA}
ANCOVA will statistically partition the effect of the covariate measure from the relationship between the treatments and the dependent variable.\\

The ANCOVA F-test is more likely to identify a statistically significant treatment effect than the ANOVA F-test.

\begin{itemize}
  \item ANOVA error term is based on variation of Y around individual group means. 
  \begin{align}
    \epsilon = 
  \end{align}
  error for subject 
  \item ANCOVA error term is based on variation of Y scores around pooled within group regression lines
\end{itemize}



%%% Local Variables: 
%%% mode: latex
%%% TeX-master: "MasterThesisSfS"
%%% End:  
%%%%%%%%%%%%%%%%%%%%%%%%%%%%%%%%%%%%%%%%%%%%%%%%%
%%% Table of contents and list of figures and %%%
%%% tables (no need to change this usually)   %%%
%%%%%%%%%%%%%%%%%%%%%%%%%%%%%%%%%%%%%%%%%%%%%%%%%
\newpage
\tableofcontents
\newpage
\listoffigures
\newpage
\listoftables

%% Notations and glossary (optional)
\cleardoublepage
\phantomsection
\addcontentsline{toc}{chapter}{\protect\numberline{}{Notation}}
\markboth{Notation}{Notation}
\include{tex/Notation}

\cleardoublepage
\pagenumbering{arabic}%--- switch back to standard numbering


%%%%%%%%%%%%%%%%%%%%%%%%%%%%%%%%%%%%%%%%%%%%%%%%%
%%% Your text... Either write here directly,  %%%
%%% or even better: write in separate files   %%%
%%% that you just have to include here.       %%%
%%%%%%%%%%%%%%%%%%%%%%%%%%%%%%%%%%%%%%%%%%%%%%%%%
$body$

%%%%%%%%%%%%%%%%%%%%%%%%%%%%%%%%%%%%%%%%%%%%%%%%%
%%% Bibliography                              %%%
%%%%%%%%%%%%%%%%%%%%%%%%%%%%%%%%%%%%%%%%%%%%%%%%%
\addtocontents{toc}{\vspace{.5\baselineskip}}
\cleardoublepage
\phantomsection
\addcontentsline{toc}{chapter}{\protect\numberline{}{Bibliography}}
\bibliography{bib/bib}

%% All books from our library (SfS) are already in a BiBTeX file
%% (Assbib). You can use Assbib combined with your personal BiBTeX file:
%% \bibliography{Myreferences,Assbib}. Of course, this will only work on
%% the computers at SfS, unless you copy the Assbib file
%%  --> /u/sfs/bib/Assbib.bib

%%%%%%%%%%%%%%%%%%%%%%%%%%%%%%%%%%%%%%%%%%%%%%%%%
%%% Appendices (if needed)                    %%%
%%%%%%%%%%%%%%%%%%%%%%%%%%%%%%%%%%%%%%%%%%%%%%%%%
\addtocontents{toc}{\vspace{.5\baselineskip}}
\appendix
\include{tex/Appendix1}
\include{tex/Appendix2}


%% Epilogue (optional)
\addtocontents{toc}{\vspace{.5\baselineskip}}
\cleardoublepage
\phantomsection
\addcontentsline{toc}{chapter}{\protect\numberline{}{Epilogue}}
\markboth{Epilogue}{Epilogue}
\include{tex/Epilogue}


%%%%%%%%%%%%%%%%%%%%%%%%%%%%%%%%%%%%%%%%%%%%%%%%%%
%%% Declaration of originality (Do not remove!)%%%
%%%%%%%%%%%%%%%%%%%%%%%%%%%%%%%%%%%%%%%%%%%%%%%%%%
%% Instructions:
%% -------------
%% fill in the empty document confirmation-originality.pdf electronically
%% print it out and sign it
%% scan it in again and save the scan in this directory with name
%% confirmation-originality-scan.pdf
%%
%% General info on plagiarism:
%% https://www.ethz.ch/students/en/studies/performance-assessments/plagiarism.html
\cleardoublepage
\includepdf[pages={-}, frame=true,scale=1]{pdf/confirmation-originality.pdf}
\end{document}
